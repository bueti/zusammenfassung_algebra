\chapter{Ganze Zahlen} % (fold)
\label{cha:ganze_zahlen}
Seinen \(x,y\) ganze Zahlen. Es gilt \(x < y\) genau dann wenn es eine natürliche Zahl \(n > 0\) mit der Eigenschaft \(x+n=y\) gibt.
\section{Teilbarkeit} % (fold)
\label{sec:teilbarkeit}
Sind \(x,y \in \mathbb{Z}\) ganze Zahlen so sagen wir, dass \emph{x eine Teiler von y} ist falls es ein \(k \in \mathbb{Z}\) gibt mit \(xk = y\). Wir schreiben in diesem Fall \(x|y\). Es gilt also:
\[ x | y : \Leftrightarrow \exists k \in \mathbb{Z} (y=xk)\]
Die Teilbarkeitsrelation ist \underline{transitiv}, d. h. wenn für beliebige ganze Zahlen \(x,y,z\) folgt aus \(x|y\) und \(y|z\) stets auch \(x|z\).

\textbf{Definition:} Zwei ganze Zahlen heissen \emph{teilerfremd} wenn \emph{ggT(x,y)}=1 gilt.\newline
Seien \(x,y \in \mathbb{Z}\) teilerfremd, dann gibt es ganze Zahlen \(k, k'\) so, dass
\[ 1 = kx + k'y \]
gilt.
% section teilbarkeit (end)

\section{Primzahlen} % (fold)
\label{sec:primzahlen}
Folgende Aussagen sind für \(p \in \mathbb{N}\) äquivalent:
\begin{enumerate}
	\item \( \forall{n,m} \in \mathbb{N} (p|nm \Rightarrow p|n \vee p|m)\) und \(p \neq 1\)
	\item \( T(p) = \{1,p\}\) und  \(p \neq 1\)
	\item \(|T(p)| = 2\)
\end{enumerate}
Weiter gilt:\newline
Mit Ausnahme der Zahl 2 sind alle Primzahlen p ungerade.

\subsection{Primfaktorzerlegung} % (fold)
\label{sub:primfaktorzerlegung}
Vorgehen: Die gegebene Zahl Modulo die kleinste, noch nicht getestete Primzahl. Falls der Rest 0 ist, weiter mit dem Ergebins, ansonsten die nächst grössere Primzahl verwenden.\newline
\textbf{Beispiel:} pfz(45)
\begin{align*}
	45 \mod 2 &= 1 \rightarrow \text{ Rest ungleich 0, nächste Primzahl } \\
	45 \mod 3 &= 0 \rightarrow 3 \cdot 15 = 45 \\
	15 \mod 3 &= 0 \rightarrow 3 \cdot 5 = 15 \\
	5 \mod 3 &= 2  \rightarrow \text{ Rest ungleich 0, nächste Primzahl } \\
	5 \mod 5 &= 0 \rightarrow 1 \cdot 5 = 5
\end{align*}
Daraus folgt, die Primfaktoren für 45 heissen: \( 3^2 \cdot 5^1 \)\newline
Um die Anzahl Primfaktoren zu bestimmen wird nur die Anzahl unterschiedlicher Basen gezählt und nicht die Anzahl der Faktoren. \textbf{45 hat demzufolge zwei Primfaktoren!}
% subsection primfaktorzerlegung (end)
% section primzahlen (end)

\section{Modulare Arithmetik} % (fold)
\label{sec:modulare_arihmetik}

% section modulare_arihmetik (end)
% chapter ganze_zahlen (end)