\chapter{Algebraische Strukturen}
Eine algebraische Struktur ist eine Menge von Mengen (die der Struktur zugrundeliegenden Mengen) die jeweils mit einer oder mehreren (meist binären) Verknüpfung versehen sind.
\section{Grundstrukturen}
\subsection{Halbgruppen, Gruppen und Monoide}
\begin{definition}
Eine Struktur $(G, \cdot)$ bestehend aus einer Menge $G$ und einer Verknüpfung $\cdot : G \times G \rightarrow G$ heisst:
\begin{enumerate}
	\item \textbf{Halbgruppe}, falls $\cdot$ assoziativ d.h. $a \cdot ( b \cdot c) = (a \cdot b) \cdot c$ ist.
	\item \textbf{Monoid}, falls $(G, \cdot)$ eine Halbgruppe ist und ein neutrales Element $e \in G$ existiert.
	\item \textbf{Gruppe}, falls $(G, \cdot)$ ein Monoid (mit neutralem Element $e$) ist und für alle $a \in G$ ein $b \in G$ existiert, so dass $a \cdot b = b \cdot a = e$ gilt.
	\item \textbf{kommutative Gruppe}, falls $(G, \cdot)$ eine Gruppe und $\cdot$ kommutativ ist.
\end{enumerate}
\end{definition}

\begin{bem}
In einer Gruppe $(M, \cdot)$ besitzt jedes Element $a \in G$ ein eindeutig bestimmtes inverses Element (dies kann auch $a$ selbst sein), wir bezeichnen dieses mit $a^{-1}$. Offensichtlich gilt für jedes $a \in G$ auch $(a^{-1})^{-1} = a$.
\end{bem}

\subsection{Unterstukturen}
Eine unter $\cdot$ abgeschlossene Teilmenge $\bigcup$ nennen wir Unterstruktur. Für sie gelten die selben Regeln wie für die normalen Strukturen. Folgerung: Jede (Halb-) Gruppe besitzt eine kleinste Unter(halb)gruppe und jeder Monoid besitzt einen kleinsten Untermonoid, die eine gegebene Teilmenge der (Halb-) Gruppe bzw. des Monoids enthalten.

\subsection{Die Morphismen von (Halb-) Gruppen und Monoiden}
\begin{itemize}
	\item Ein \textbf{(Halb-) Gruppenhomomorphismus} von der (Halb-) Gruppe $(G, \cdot)$ in die (Halb-) Gruppe $(G', \circ)$ ist eine Abbildung $f: G \rightarrow G'$, so dass für alle $a,b \in G$ 
	$$f(a \cdot b) = f(a) \circ f(b)$$
	gilt.
	\item Ein \textbf{Monoidhomomorphismus} vom Monoid $(M, \cdot)$ in den Monoid $(M', \circ)$ ist eine Abbildung $f: M \rightarrow M'$, so dass für alle $a,b \in M$
	$$ f(a\cdot b) = f(a) \circ f(b)$$
	gilt, und ausserdem wird das neutrale Element von $(M, \cdot)$ aus das neutrale Element von $(M', \circ)$ abgebildet.
\end{itemize}
Injektive\footnote{Injektiv: Jedes Element in der Abbildung wird nur einmal erreicht} Homomorphismen nennen wir \textbf{Monomorphismen}, surjektive\footnote{Surjektiv: Es gibt Punkte in der Abbildung die mehrmals erreicht werden können} \textbf{Epimorphismen} und bijektive\footnote{Bijektiv: Es findet eine vollständige Paarbildung zwischen der Definitions- und Zielmenge statt} \textbf{Isomorphismen}.
\begin{bem}
Nicht jeder Halbgruppenhomomorphismus zwischen Monoiden ist auch ein Monoidhomomorphismus. Des weiteren gilt, dass wenn $f: (G, \cdot) \rightarrow (G', \star) $ und $h: (G', \star) \rightarrow (G'', \bullet) $ Homomorphismen von Gruppen oder Halbgruppen oder Monoiden sind, dann ist auch $h \circ f : (G, \cdot) \rightarrow (G'' \rightarrow \bullet)$ ein entsprechender Homomorphismus.
\end{bem}

\subsection{Ringe und Körper}
\begin{definition}
Eine Struktur $(R,+,\cdot)$ heisst Ring, wenn folgende Bedingungen erfüllt sind:
	\begin{enumerate}
		\item $(R, +)$ ist eine kommutative\footnote{Kommutativ: Reihenfolge egal; $a+b = b+a$} Gruppe
		\item $(R,\cdot)$ ist eine Halbgruppe
		\item Es gilt das Distributivgesetz, d. h. für alle Elemente $r,s,t$ des Rings gelten:
		\begin{enumerate}
			\item $r \cdot (s+t) = (r \cdot s) + (r \cdot t)$
			\item $(r+s) \cdot t = (r \cdot t) + (s \cdot t)$
		\end{enumerate}
	\end{enumerate}
\end{definition}
Die Beweise zu den Rechenregeln sind in den Notizen zu finden.
\begin{bsp}
\begin{tabular}{|c|c|}
\hline 
\rule[-1ex]{0pt}{2.5ex} Ringe & Keine Ringe \\ 
\hline 
\rule[-1ex]{0pt}{2.5ex} $(\mathbb{Z}, +, \cdot)$ & $(\mathbb{N}, +, \cdot)$, da $+$ auf $\mathbb{N}$ nicht kommutativ ist \\ 
\hline 
\rule[-1ex]{0pt}{2.5ex}  $(\mathbb{Z}_{/n}, +, \cdot)$ & \\ 
\hline 
\rule[-1ex]{0pt}{2.5ex}  $(\{0\}, +, \cdot)$ & \\ 
\hline 
\end{tabular} 
\end{bsp}

\subsubsection*{Nullteiler}
\begin{definition}
Ein Nullteiler ist eine Zahl, welche nicht 0 ist, und mit einer anderen Zahl multipliziert 0 ergibt.\\
Es sei $(R, +, \cdot)$ ein Ring
	\begin{enumerate}
		\item Ein Element $(r \in R)$ heisst rechter Nullteiler in $R$, falls ein $s \in R\\{0}$ existiert mit $sr = 0$.
		\item Ein Element $(r \in R)$ heisst linker Nullteiler in $R$, falls ein $s \in R\\{0}$ existiert mit $rs = 0$.
		\item Ein Element $(r \in R)$ heisst Nullteiler in $R$, falls ein $r$ sowohl linker- als auch rechter Nullteiler in $R$ ist.
		\item Der Ring $(R, +, \cdot)$ heisst Integritätsring, wenn:
		\begin{enumerate}
			\item Die Verknüpfung $\cdot$ kommutativ ist.
			\item $0 \in R$ ist der einzige Nullteiler in $R$
		\end{enumerate}
	\end{enumerate}
\end{definition}

\subsubsection*{Integritätsring}
Ein Integritätsring ist ein nullteilerfreier kommutativer Ring mit einem Einselement.
\begin{bsp} für Integritätsringe:
\begin{itemize}
	\item $\mathbb{Z}$
	\item Jeder Körper ist ein Integritätsring. Ausserdem ist jeder endliche Integritätsring ein endlicher Körper.
	\item Ein Polynomring ist ein Integritätsring, wenn die Koeffizienten aus einem Integritätsring stammen.
	\item Der Restklassenring $\mathbb{Z}_{/n}$  ist genau dann ein Integritätsring, sogar ein Körper, wenn $n$ eine Primzahl ist.
\end{itemize}
\end{bsp} 
 \begin{bem}
In einem Integritätsring $R$ gilt stets $1 \neq 0$. $1=0$ gilt nur in einem Nullring \footnote{$(\{0\}, +, \cdot)$}. Ein Nullring ist jedoch kein Integritätsring, da $0$ kein Nullteiler von $0$ ist.
\end{bem}

