\chapter{Algebraische Strukturen}
Eine algebraische Struktur ist eine Menge von Mengen (die der Struktur zugrundeliegenden Mengen) die jeweils mit einer oder mehreren (meist binären) Verknüpfung versehen sind.
\section{Grundstrukturen}
\subsection{Halbgruppen, Gruppen und Monoide}
\begin{definition}
Eine Struktur $(G, \cdot)$ bestehend aus einer Menge $G$ und einer Verknüpfung $\cdot : G \times G \rightarrow G$ heisst:
\begin{enumerate}
	\item \textbf{Halbgruppe}, falls $\cdot$ assoziativ d.h. $a \cdot ( b \cdot c) = (a \cdot b) \cdot c$ ist.
	\item \textbf{Monoid}, falls $(G, \cdot)$ eine Halbgruppe ist und ein neutrales Element $e \in G$ existiert.
	\item \textbf{Gruppe}, falls $(G, \cdot)$ ein Monoid (mit neutralem Element $e$) ist und für alle $a \in G$ ein $b \in G$ existiert, so dass $a \cdot b = b \cdot a = e$ gilt.
	\item \textbf{kommutative Gruppe}, falls $(G, \cdot)$ eine Gruppe und $\cdot$ kommutativ ist.
\end{enumerate}
\end{definition}

\begin{bem}
In einer Gruppe $(M, \cdot)$ besitzt jedes Element $a \in G$ ein eindeutig bestimmtes inverses Element (dies kann auch $a$ selbst sein), wir bezeichnen dieses mit $a^{-1}$. Offensichtlich gilt für jedes $a \in G$ auch $(a^{-1})^{-1} = a$.
\end{bem}

\subsection{Unterstukturen}
Eine unter $\cdot$ abgeschlossene Teilmenge $\bigcup$ nennen wir Unterstruktur. Für sie gelten die selben Regeln wie für die normalen Strukturen. Folgerung: Jede (Halb-) Gruppe besitzt eine kleinste Unter(halb)gruppe und jeder Monoid besitzt einen kleinsten Untermonoid, die eine gegebene Teilmenge der (Halb-) Gruppe bzw. des Monoids enthalten.

\subsection{Die Morphismen von (Halb-) Gruppen und Monoiden}
