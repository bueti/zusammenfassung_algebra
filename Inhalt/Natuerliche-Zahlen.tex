\chapter{Natürliche Zahlen} % (fold)
\label{cha:natürliche_zahlen}

\section{Die grundlegende Struktur der natürlichen Zahlen} % (fold)
\label{sec:die_grundlegende_struktur_der_natürlichen_zahlen}
\textbf{Definition:} 
\begin{enumerate}
	\item Jede natürliche Zahl \(k\) hat genau einen Nachfolger \(N(k)\).
	\item 0 ist kein Nachfolger aber alle anderen natürlichen Zahlen sind Nachfolger von genau einer natürlichen Zahl.
	\item Ist \(X \subset \mathbb{N}\) mit \(0 \in X\) eine Menge von natürlichen Zahlen mit der Eigenschaft, dass für jedes Element \(k\) von \(X\) auch \(N(k)\) zu \(x\) gehört, dann ist \(X = \mathbb{N}\).
\end{enumerate}
Die letzte der oben genannten Eigenschaften wird das Prinzip der vollständigen Induktion gennant.
\subsection{Vollständige Induktion} % (fold)
\label{sub:vollständige_induktion}
Zu Beweisen:\[\sum_{i=1}^n i = \frac{n(n+1)}{2} \]
Induktionsverankerung
\begin{align*}
	\text{für n=1: } \sum_{i=1}^1 i = 1 = \frac{2}{2} = \frac{1\cdot2}{2} = \frac{1(1+1)}{2}
\end{align*}
Induktionsschritt: Für \(n \rightarrow n+1\) \newline
Induktionsannahme:
\[ \sum_{i=1}^{n+1} i = \frac{(n+1)((n+1)+1)}{2} \]
\begin{align*}
	\sum_{i=1}^{n+1} i &= \sum_{i=1}^{n} i + (n+1) \\
	&=^{iA} \frac{n(n+1)}{2} + (k+1) \\
	&= \frac{n(n+1)}{2} +  \frac{2n(n+1)}{2} \\
	&= \frac{n(n+1)+2(n+1)}{2} \\
	&= \frac{(k+1)(k+2)}{2} \\
	\sum_{i=1}^{n+1} i &= \frac{(n+1)((n+1)+1)}{2}
\end{align*}
% subsection vollständige_induktion (end)
% section die_grundlegende_struktur_der_natürlichen_zahlen (end)

\section{Rekursive Definitionen} % (fold)
\label{sec:rekursive_definitionen}
Ist \(M\) eine beliebige Menge und \(g: M \rightarrow M\) sowie \(c \in M\), dann gibt es eine eindeutig bestimmte Funktion \(f: \mathbb{N} \rightarrow M\) welche die Gleichungen
\begin{align*}
	f(0) &= c \\
	f(k+1) &= f(g(k))
\end{align*}
erfüllt.
% section rekursive_definitionen (end)

\section{Die algebraische Struktur der natürlichen Zahlen} % (fold)
\label{sec:die_algebraische_struktur_der_natürlichen_zahlen}
Siehe Skript...
% section die_algebraische_struktur_der_natürlichen_zahlen (end)

\section{Ordnungstheoretische Struktur der natürlichen Zahlen} % (fold)
\label{sec:ordnungstheoretische_struktur_der_natürlichen_zahlen}
Nicht wichtig, siehe Skript
% section ordnungstheoretische_struktur_der_natürlichen_zahlen (end)
% chapter natürliche_zahlen (end)