\chapter{Mengen}
\section{Allgemein}
Eine Menge ist die Zusammenfassung von (mathematischen) Objekten zu einem neuen Ganzen, welches für sich selbst genommen wieder ein mathematisches Objekt darstellt. Weiter gelte das \emph{Prinzip der extensionalen Gleichheit}, welches wie folgt lautet:
\par \hspace*{10mm} \emph{Zwei Mengen sind genau dann gleich, wenn sie dieselben Elemente enthalten.}

\begin{longtable}{|p{0.2\textwidth}|p{0.3\textwidth}|p{0.5\textwidth}|}
	\hline
	\( A \in B \) & A ist Element von B & \\
	\hline
	\( A \subset B \) & A ist Teilmenge von B & Jedes Element von A kommt in B vor \\
	\hline
	\( A \subseteq B \) & echte Teilmenge & Wenn \(A \subset B \) und \(A \neq B \) ist \\
	\hline
	\( A \cap B \) & Schnittmenge & Alle Elemente die zu A sowie zu B gehören \\
	\hline
	\( A \cup B \) & Vereinigungsmenge & Alle Elemente die zu A oder zu B oder zu beiden gehören \\
	\hline
	\( A \setminus B \) & Differenzmenge & Alle Elemente die zu A aber nicht zu B gehören \\
	\hline
	\( \emptyset \) & Leeremenge & \\
	\hline
	\( \powerset(A) \) & Potenzmenge & \( \powerset(A) := \{x|x \subset A\} \) \\
	\hline
\end{longtable}

\section{Rechenregeln}
\begin{enumerate}
	\item Kommutativität der Vereinigung und des Schnittes:
		$$ A \cup B = B \cup A \text{ und } A \cap B = B \cap A$$
	\item Assoziativität: 
		$$ A \cap (B \cap C) = (A \cap B) \cap C  \text{ und }  A \cup (B \cup C) = (A \cup B) \cup C) $$
	\item Distributivität:
		$$ A \cap (B \cup C) ? (A \cap B) \cup (A \cap C)  \text{ und } a \cup (B \cap C) = (A \cup B) \cap (A \cup C)$$
	\item Idempotenz: 
		$$ A \cap A = A  \text{ und }  A \cup A = A$$
	\item De Morgan
	$$(C\setminus A)\cap (C\setminus B) = C\setminus (A \cup B)  \text{ und } (C\setminus A) \cup (C \setminus B) = C\setminus (A \cap B)$$
\end{enumerate}
\section{Mengenbildung}
\subsection{Leeremenge}
Die Leeremenge ist Teilmenge von jeder Menge da jedes Element der Leermenge Teil jeder Menge ist.
\subsection{Vereinigung}
Ist A eine Menge von Mengen, dann definieren wir die \emph{Vereinigung} \( \bigcup A\) von A als die Menge, welche alle Dinge enthält die ein Element eines Elementes von A sind.
\par \hspace*{10mm} \( x \in \cup A \Leftrightarrow \exists Y \in A(x \in Y) \).
\newline A ist eine Menge von Mengen: \newline
\hspace*{10mm}\( \cup A = \{x | \forall_X \in A (x \in X)\}\) \newline
\hspace*{10mm}\( \cup \{A, B, C\} = A \cup B \cup C = \{x|x \in A \vee x \in B \vee x \in C\}\) \newline
\hspace*{10mm}\( X \cup Y = \{ x | x \in Y \vee x \in X \}\)
\subsection{Schnittmenge}
Ist A eine nicht leere Menge von Mengen, dann definieren wir die Schnittmenge \(\bigcap A\) von A, als die Menge die alle Dinge enthält, die ein Element von jedem Element von A sind.
\par \hspace*{10mm} \( x \in \cap A \Leftrightarrow \forall Y \in A(x \in Y) \).
\subsection{Potenzmenge}
die Potenzmenge von A ist die Menge aller Teilmengen inkl. der Leerenmenge von A. \newline
\( \powerset(\emptyset) = \{\emptyset\} \neq \emptyset \) \newline
\( \powerset(\{0, 1\}) = \{\emptyset, \{0\}, \{1\}, \{0,1\}\} \)
